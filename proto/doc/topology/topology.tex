%        File: topology.tex
%     Created: Thu Nov 09 01:00 PM 2006 C
% Last Change: Thu Nov 09 01:00 PM 2006 C
%
\documentclass[a4paper]{article}
\usepackage{color,graphicx}
\usepackage{amsmath}
\usepackage{amsthm}
\usepackage{amssymb}
\usepackage{amsfonts}
\RequirePackage{ifpdf} % running on pdfTeX?
\ifpdf
\usepackage[pdftex]{hyperref}
\else
\newcommand{\href}[2]{ #1 }
\fi
\title{Netsukuku topology}
\author{http://netsukuku.freaknet.org/}
\begin{document}
\maketitle

\begin{abstract}
	This document describes the structure of the network topology of
	Netsukuku and its practical implementation.
	%%% TODO: aggiungi qualcosa in piu'
\end{abstract}

\section{Preface}
\label{sec:preface}

We're assuming that you already know the basics of the QSPN. If not, read the
QSPN document first: \cite{qspndoc}.

\section{The general idea}
\label{sec:general_idea}

The aim of Netsukuku is to be a (physical) scalable mesh network, completely
distributed and decentralised, anonymous and autonomous.

The software, which must be executed by every node of the net, has to be
unobtrusive. It has to use very few CPU and memory resources, in this way it
will be possible to run it inside low-performance computers, like Access Points,
embedded devices and old computers.

If this requirements are met, Netsukuku can be easily used to build a worldwide
distributed, anonymous and not controlled network, separated from the
Internet, without the support of any servers, ISPs or control authorities.

\section{Basic definitions}

\begin{description}
	\item[Node] We call \emph{node} any computer that is hooked up to the
		Netsukuku network.
	\item[Rnode] stands for remote node: given a node X, it is any other
		node directly linked to X, i.e. it's a neighbour of X.
	\item[Map] A map is a file, keeped by each node, whichs contains all the
		necessary information about the network, f.e. routes and nodes
		status.
\end{description}
Example:\\
\begin{figure}[h]
	\begin{center}
		\includegraphics[scale=0.5]{fig/segABC}
	\end{center}
	\caption{The nodes A,B and C}
\end{figure}
A is the rnode of B.\\
B is the rnode of A and C.\\
C is the rnode of B.

\section{Network topology}
\label{sec:net_topology}

A simple topology, which doesn't impose any structure on the network, can be
memorised with a simple map. In this map, all the information regarding the
nodes of the network have to be memorized. Surely, this kind of map cannot be
utilised by Netsukuku, because it would require too much memory.
For example, even if we store just one route to reach one node and even if
this route costs one byte, we would need 1Gb of memory for a network composed
by $10^9$ nodes (the current Internet).

For this reason, it's necessary to structure the network in a convenient
topology.

\subsection{Fractal topology}
\label{sec:fractal_topology}
\subsubsection{Level 1}
First of all we'll subdivide the network in groups of 256 nodes and we'll use
the following definitions:
\begin{description}
	\item[Gnode] means group node. It is a group of nodes, i.e. a set of
		nodes. Each node of the network belongs to just one gnode.\\
		A gnode contains a maximum of 256 nodes.
	\item[Bnode] stands for border node. It is a node which belongs to a
		gnode G, but that is also diretcly linked to at least one node
		of another gnode, i.e. some of its rnodes belongs to different
		gnodes than its.
\end{description}

Example:\\
\begin{figure}[h]
	\begin{center}
		\includegraphics[scale=0.5]{fig/bnode}
	\end{center}
	\caption{The bnode A and B, belonging respectively to the gnode G and
	$G'$}
\end{figure}
A is a node belonging to the gnode G, its rnode is B.\\
B is a node belonging to the gnode $G'$, its rnode is A.\\
A is a bnode of G, while B is a bnode of $G'$.

\subsubsection{Level n}
We further subdivide the network topology in \emph{groups of 256 groups of nodes}
and we continue to name them as gnode.\\
At this point, we repeat recursively this subdivision process until
we can group all the nodes of the network into a single gnode.

Doing so, we've structured the network in $n$ levels (from $0$ to $n-1$).\\
In the base level (level 0), there are 256 single nodes.\\
In the first level (level 1), there are 256 normal gnodes. Each of them
contains 256 single nodes.\\
In the second (level 2), 256 gnodes of level 1 forms a single \emph{group of
groups of nodes}.\\
In the third (level 3), there are 256 groups of 256 groups of 256 groups of
256 nodes.\\
Continuing in this way, we arrive at the last level (level $n-1$), where there
is a single group which contains the whole network.\\

The QSPN algorithm is able to operate independently on any level,
considering each gnode as a single node of level 0.
Hence, we've achieved a fractal topology.

\subsubsection*{Example}

Figure \ref{fig:fract_circle}\footnote{this figure has been taken from:
\href{http://www.ian.org/FX/Plugins.html}{http://www.ian.org/FX/Plugins.html}}
is an example of the fractal topology of Netsukuku.

\begin{figure}[h]
	\begin{center}
		\includegraphics[scale=0.5]{fig/fractal_circle}
	\end{center}
	\caption{An example of the netsukuku topology structure}
	\label{fig:fract_circle}
\end{figure}

In this topology, each gnode contains four nodes, i.e. each group contains
four elements. The network is structured in 5 levels. The red elements, are single
nodes. The bright green circle are groups of nodes. The dark green circles are
groups of groups of nodes. The dark blue circle are groups of groups of groups of
nodes. Finally, the bright blue circle is the gnode which contains the whole
network.


%%%% TODO: organizzazione della mappa, a ogni gnode viene assegnato un ID,
%%%%       un nodo non puo' vedere la mappa interna dei gnode a cui non
%%%%      appartiene.


%%%%%%%%%%%%%%%%%%%%%% TODO %%%%%%%%%%%%%%%%%%%%%
%TODO: CIDR routing

%bnode map
%	Size
%ext map  (sizeof)
%int map	(sizeof)
%IP addresses
%QSPN per livelli alti: chi manda chi e cosa


%%%%%%%%%%%%%%%%
% Bibliography %
%%%%%%%%%%%%%%%%

\begin{thebibliography}{99}
	\bibitem{qspndoc} QSPN document:
		\href{http://netsukuku.freaknet.org/doc/main\_doc/qspn.pdf}{qspn.pdf}
	\bibitem{ntksite} Netsukuku website:
		\href{http://netsukuku.freaknet.org/}{http://netsukuku.freaknet.org/}
\end{thebibliography}
\newpage

\begin{center}
\verb|^_^|
\end{center}
\end{document}
